\documentclass[a4paper]{article}
\usepackage{longtable}
\usepackage[color]{vdmlisting}
\usepackage{fullpage}
\usepackage{hyperref}
\begin{document}
\title{}
\author{}
\begin{vdm_al}

class Transaction
types
    public Type = <Deposit> | <Withdrawal>;
instance variables
    public amount: real;
    public type : Type; 
    public id : nat; --Id will be the cardinality of transactions in a class, e.g 2 classes can both have transactions with id 1
    --note : seq of char?
operations
    public Transaction: real * Type * nat ==> Transaction
    Transaction(val,t_type,i) == (
        amount := val;
        type := t_type;
        id := i;
    )
    pre val > 0
    post type = <Deposit> or type = <Withdrawal>;

    public getAmount: () ==> nat1
    getAmount() == (
        return amount;
    );
    public getType: () ==> Type
    getType() == (
        return type;
    );
    public getId: () ==> nat
    getId() == (return id;);
end Transaction
\end{vdm_al}
\bigskip
\begin{longtable}{|l|r|r|}
\hline
Function or operation & Coverage & Calls \\
\hline
\hline
Transaction & 100.0\% & 10 \\
\hline
Transaction & 100.0\% & 10 \\
\hline
getAmount & 100.0\% & 4 \\
\hline
getId & 100.0\% & 1 \\
\hline
getType & 100.0\% & 4 \\
\hline
\hline
transaction.vdmpp & 100.0\% & 29 \\
\hline
\end{longtable}
\end{document}
