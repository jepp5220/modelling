\documentclass[a4paper]{article}
\usepackage{longtable}
\usepackage[color]{vdmlisting}
\usepackage{fullpage}
\usepackage{hyperref}
\begin{document}
\title{}
\author{}
\begin{vdm_al}
--  Overture STANDARD LIBRARY: VDM-Unit
--      --------------------------------------------
-- Version 2.1.0 
--
-- Standard library for the Overture Interpreter. When the interpreter
-- evaluates the preliminary functions/operations in this file,
-- corresponding internal functions is called instead of issuing a run
-- time error. Signatures should not be changed, as well as name of
-- module (VDM-SL) or class (VDM++). Pre/post conditions is 
-- fully user customisable. 
-- Dont care's may NOT be used in the parameter lists.
--  
--
-- define VDM++ equivalents to the JUnit 3.8.2 classes
--
--===========================================================================
-- Using VDMUnit
--
-- VDM Unit generally supports the same features as JUnit for Java
--
-- Approach:
-- * Define one Test Case per class which should be tested
-- ** Specify all test *operations* in the test case
-- * Define a TestResult and TestSuite to execute the tests
--
-- Running the tests:
-- Option 1.
-- Use the automated reflection search to find Test classes.
--  new TestRunner().run()
--
-- Option 2.
-- Use the automated reflection search to find test operations.
--  All operations must begin with "test".
--  Usage:
--      let tests : set of Test = {new TestCase1(), new TestCase2()},
--          ts : TestSuite = new TestSuite(tests),
--          result : TestResult = new TestResult()
--      in
--      (
--          ts.run(result);
--          IO`print(result.toString());
--      );
--
-- Option 3.
-- Use explicit definition to call tests.
--  All test cases must *override* the operation 
--      "protected runTest : () ==> ()"
--  This means that only a single test can be performed in each test case
--
-- Usage:
--      let ts : TestSuite = new TestSuite(),
--          result = new TestResult() 
--      in
--      (
--          ts.addTest(new TestCase1());
--          ts.addTest(new TestCase2());
--          ts.run(result);
--          IO`println(result.toString());
--      )
--
--===========================================================================



-- forward definition of java.lang.Throwable
-- do not code generate (but import from Java library)
class Throwable
end Throwable

-- forward definition of java.lang.Error
-- do not code generate (but import from Java library)
class Error is subclass of Throwable

instance variables
  protected fMessage : seq of char := []
  
operations
  public Error: () ==> Error
  Error () == skip;
  
  public Error: seq of char ==> Error
  Error (pmessage) == fMessage := pmessage;

  public hasMessage: () ==> bool
  hasMessage () == return len fMessage > 0;
    
  public getMessage: () ==> seq of char
  getMessage () == return fMessage
    pre len fMessage > 0;--hasMessage()


  public static throw: seq of char ==> ()
  throw (pmessage) == exit new Error(pmessage)
end Error

--
-- forward definition of junit.framework.AssertionFailedError
--

class AssertionFailedError is subclass of Error

operations
  public AssertionFailedError: () ==> AssertionFailedError
  AssertionFailedError () == skip;
  
  public AssertionFailedError: seq of char ==> AssertionFailedError
  AssertionFailedError (pmessage) == fMessage := pmessage;
   
end AssertionFailedError

--
-- forward definition of junit.framework.Assert
--

class Assert

operations
  public static assertTrue: bool ==> ()
  assertTrue (pbool) ==
    if not pbool then exit new AssertionFailedError();
  
  public static assertTrue: seq of char * bool ==> ()
  assertTrue (pmessage, pbool) ==
    if not pbool then exit new AssertionFailedError(pmessage);
    
  public static assertFalse: bool ==> ()
  assertFalse (pbool) ==
    if pbool then exit new AssertionFailedError();
  
  public static assertFalse: seq of char * bool ==> ()
  assertFalse (pmessage, pbool) ==
    if pbool then exit new AssertionFailedError(pmessage);
    
  public static fail: () ==> ()
  fail () == exit new AssertionFailedError();
  
  public static fail: seq of char ==> ()
  fail (pmessage) == exit new AssertionFailedError(pmessage)
    
end Assert

--
-- forward definition of junit.framework.Test
--

class Test is subclass of Assert

instance variables
  private fName : seq of char := []
  
operations
  -- instance variables access operations
  public getName: () ==> seq of char
  getName () == return fName;
  
  public setName: seq of char ==> ()
  setName (name) == fName := name;
  
  public run: TestResult ==> ()
  run (-) == is subclass responsibility;
  
  public runOnly: seq of char * TestResult ==> ()
  runOnly (-,-) == is subclass responsibility
    
end Test

--
-- forward definition of junit.framework.TestCase
--

class TestCase is subclass of Test

operations

--  /**
--   * A convenience method to run this test, collecting the results with a
--   * default TestResult object.
--   *
--   * @see TestResult
--   */
    public run :() ==> TestResult
    run() ==
    (
        let result : TestResult = createResult() in (
        run(result);
        return result);
    );
--  /**
--   * Runs the test case and collects the results in TestResult.
--   */
    public run : TestResult ==> ()
    run(result) ==
    (
        result.run(self);
    );
 
 public runOnly: seq of char * TestResult ==> ()
 runOnly (name,result) ==
 if name = getName() then
  result.run(self);
  
    protected createResult : () ==> TestResult
    createResult() == return new TestResult();
 
operations
  -- constructor
  public TestCase: seq of char ==> TestCase
  TestCase (name) == setName(name);

  public TestCase: () ==> TestCase
  TestCase () == skip;
    
  -- template method pattern
  public setUp: () ==> ()
  setUp () == skip;
  
  --
  -- This method might be overriden in a sub class or it will
  --  use reflection to call the test method as specified in the name
  --
  protected runTest: () ==> ()
  runTest () == reflectionRunTest(self,getName());

  --
  -- External Method
  --
  private static reflectionRunTest : TestCase * seq of char ==> ()
  reflectionRunTest(obj,name) == is not yet specified;
     
  public runBare : () ==> ()
  runBare() ==
  (
    dcl exception : [Throwable] := nil;
    
    setUp();
    
    trap exc :Throwable
        with  
            exception := exc
        in
        (
            runTest();
        );

    trap exc :Throwable
        with  
            if exception = nil then
                exception := exc
        in
        (
            tearDown();
        );
        
    if exception <> nil then exit exception;
   );
  
  public tearDown: () ==> ()
  tearDown () == skip;
           
end TestCase

--
-- forward definition of junit.framework.TestSuite
--

class TestSuite is subclass of Test

instance variables
  private fTests : seq of Test := []
  
operations
  public TestSuite: seq of char ==> TestSuite
  TestSuite (name) == setName(name);

  public TestSuite: () ==> TestSuite
  TestSuite () == setName("");

  public TestSuite: set of Test * seq of char ==> TestSuite
  TestSuite (tests,name) == 
  (
    setName(name);
    for all test in set tests do
    (
       for all t in set elems createTests(test) do
       (
         addTest(t); 
       );
    );
    
  );

  public TestSuite: set of Test ==> TestSuite
  TestSuite (tests) == 
  (
    setName("");
    for all test in set tests do
    (
       for all t in set elems createTests(test) do
       (
         addTest(t); 
       );
    );
    
  );

  public TestSuite: Test ==> TestSuite
  TestSuite (test) == 
  (
    setName("");
    for all t in set elems createTests(test) do
     (
       addTest(t); 
     );
  );

  public TestSuite: Test * seq of char ==> TestSuite
  TestSuite (test,name) == 
  (
    setName(name);
    for all t in set elems createTests(test) do
     (
       addTest(t); 
     );
  );
  
  public addTest: Test ==> ()
  addTest (ptst) == fTests := fTests ^ [ptst];
  
  -- implementing the abstract run operation 
  public run: TestResult ==> ()
  run (ptr) ==for ptst in fTests do ptst.run(ptr);

  public runOnly: seq of char * TestResult ==> ()
  runOnly (name,ptr) ==
  for ptst in fTests do
  if ptst.getName() = name then
 ptst.run(ptr);
  
  public run: Test * TestResult ==> ()
  run (test, ptr) == test.run(ptr);
      
    public tests : () ==> seq of Test
    tests()==return fTests;
    
    public testCount : () ==> int
    testCount()== return len fTests;
    
    public testAt : int ==> Test
    testAt(index) == return fTests(index)
    pre index >0 and index < len fTests;


  private getTestMethodNamed : Test ==> seq of seq of char
  getTestMethodNamed(test)== is not yet specified;

  public createTests : Test ==> seq of Test
  createTests(test)== is not yet specified;


end TestSuite

--
-- forward definition of junit.framework.TestListener
--

class TestListener

operations
  public initListener: () ==> ()
  initListener () == is subclass responsibility;
  
  public exitListener: () ==> ()
  exitListener () == is subclass responsibility;
  
  public addFailure: Test * AssertionFailedError ==> ()
  addFailure (-, -) == is subclass responsibility; 

  public addError: Test * Throwable ==> ()
  addError (-, -) == is subclass responsibility;
  
  public startTest: Test ==> ()
  startTest (-) == is subclass responsibility;
  
  public endTest: Test ==> ()
  endTest (-) == is subclass responsibility;
  
end TestListener

--
-- forward definition of junit.framework.TestResult
--

class TestResult

types
  private InternalError :: tcname : seq of char
                           except : Error
                   
instance variables
  -- keep track of assertions failed
  private fFailures : seq of AssertionFailedError := [];
  
  -- keep track of exceptions thrown
  private fErrors : seq of InternalError := [];
  
  -- count the number of executed tests
  private fRunTests : nat := 0;
  
  -- the list of interested listeners
  private fListeners : set of TestListener := {}
  
operations
  public addListener: TestListener ==> ()
  addListener (ptl) == 
    ( -- first add the listener to the set
      fListeners := fListeners union {ptl};
      -- execute the initializer for the listener
      ptl.initListener() )
    pre ptl not in set fListeners;
  
  public removeListener: TestListener ==> ()
  removeListener (ptl) ==
    ( -- execute the exit operation for the listener
      ptl.exitListener();
      -- remove the listener from the set
      fListeners := fListeners \ {ptl} )
    pre ptl in set fListeners;
  
  public addFailure: Test * AssertionFailedError ==> ()
  addFailure (ptst, pafe) == 
    ( -- first handle the failed assertion locally
      if pafe.hasMessage()
      then fFailures := fFailures ^ [pafe]
      else let str = "Assertion failed in test "^ptst.getName() in
             fFailures := fFailures ^ [new AssertionFailedError(str)];
      -- now inform all listeners
      for all listener in set fListeners do
        listener.addFailure(ptst, pafe) );
  
  public addError: Test * Throwable ==> ()
  addError (ptst, perr) ==
    ( -- add the error to the list of exceptions caught so far
      fErrors := fErrors ^ [mk_InternalError(ptst.getName(), perr)];
      -- now inform all listeners
      for all listener in set fListeners do
        listener.addError(ptst, perr) );

  public startTest: Test ==> ()
  startTest (ptst) ==
    ( -- first increase the local run test counter
      fRunTests := fRunTests + 1;
      -- now inform all listeners
      for all listener in set fListeners do
        listener.startTest(ptst) );
  
  public endTest: Test ==> ()
  endTest (ptst) ==
    for all listener in set fListeners do
      listener.endTest(ptst);

  public run : TestCase ==> ()
  run(test)==
  ( 
    startTest(test);
    runProtected(test);
    endTest(test);

  );
  public runProtected : TestCase ==> () 
  runProtected(test) ==
  (
    trap exc :Throwable
        with
          if isofclass(AssertionFailedError,exc)
          then addFailure(test, exc)
          else if isofbaseclass(Throwable, exc)
               then addError(test, exc)
               else error -- We hope this never happens
        in (
              test.runBare();
            ); 
  );
  
  public runCount: () ==> nat
  runCount () == return fRunTests;
    
  public failureCount: () ==> nat
  failureCount () == return len fFailures;
  
  public errorCount: () ==> nat
  errorCount () == return len fErrors;
  
  public wasSuccessful: () ==> bool
  wasSuccessful () == return failureCount() = 0 and errorCount() = 0;

  public toString : () ==> seq of char
  toString()==
  (
    dcl tmp: seq of char :=     
        "----------------------------------------\n"^
        "|    TEST RESULTS                      |\n"^
        "|--------------------------------------|\n"^
        "| Executed: "^ToStringInt(runCount())^"\n"^
        "| Failures: "^ToStringInt(failureCount())^"\n"^
        "| Errors:   "^ToStringInt(errorCount())^"\n";
    
    tmp := tmp^"|______________________________________|\n|";
    for all i in set inds fFailures do
        tmp := tmp^"\n| "^fFailures(i).getMessage();
    
    tmp := tmp^"\n|";

    for all i in set inds fErrors do
        tmp := tmp^"\n| Error in test "^fErrors(i).tcname^" "^fErrors(i).except.getMessage();
 
    tmp := tmp^"\n|--------------------------------------|\n";

    if wasSuccessful() 
    then
    (
        tmp := tmp^"|              SUCCESS                 |";
    )
    else
    (
        tmp := tmp^"|              FAILURE                 |";
    );
        tmp := tmp^"\n|______________________________________|";
        return tmp;
  );
functions
-------------------------------------
-- Convert a int to a String
-------------------------------------
public static ToStringInt : int | nat1 | nat -> seq of char
ToStringInt(val) ==
  let result : int = val mod 10,
      rest : int = val div 10
  in if rest > 0 then
       ToStringInt(rest) ^ GetStringFromNum(result)
     else GetStringFromNum(result)
pre val >= 0
measure ToStringIntMeasure;

private ToStringIntMeasure : nat -> nat
ToStringIntMeasure(a) == a;
    
-------------------------------------
-- Convert a int < 10 to a String
-------------------------------------
public static GetStringFromNum : int  -> seq of char --| nat | real
GetStringFromNum(val) == ["0123456789"(val+1)]
pre val < 10;
  
end TestResult  



class TestRunner

operations
public run : () ==> ()
run() ==
      let tests : set of Test = collectTests(new TestRunner()),
          ts : TestSuite = new TestSuite(tests),
          result : TestResult = new TestResult()
      in
      (
          ts.run(result);
          IO`print(result.toString());
      );

collectTests : TestRunner ==> set of Test
collectTests(obj) == is not yet specified;

end TestRunner
\end{vdm_al}
\bigskip
\begin{longtable}{|l|r|r|}
\hline
Function or operation & Coverage & Calls \\
\hline
\hline
Assert & 16.1\% & 45 \\
\hline
AssertionFailedError & 0.0\% & 0 \\
\hline
AssertionFailedError & 0.0\% & 0 \\
\hline
AssertionFailedError:106 & 0.0\% & 0 \\
\hline
Error & 0.0\% & 0 \\
\hline
Error & 0.0\% & 0 \\
\hline
Error:81 & 0.0\% & 0 \\
\hline
GetStringFromNum & 0.0\% & 0 \\
\hline
Test & 42.8\% & 8 \\
\hline
TestCase & 0.0\% & 0 \\
\hline
TestCase & 33.3\% & 0 \\
\hline
TestCase:210 & 100.0\% & 7 \\
\hline
TestListener & 0.0\% & 0 \\
\hline
TestResult & 9.0\% & 1 \\
\hline
TestRunner & 0.0\% & 0 \\
\hline
TestSuite & 0.0\% & 0 \\
\hline
TestSuite & 15.5\% & 1 \\
\hline
TestSuite:275 & 100.0\% & 1 \\
\hline
TestSuite:278 & 0.0\% & 0 \\
\hline
TestSuite:292 & 0.0\% & 0 \\
\hline
TestSuite:306 & 0.0\% & 0 \\
\hline
TestSuite:316 & 0.0\% & 0 \\
\hline
Throwable & 0.0\% & 0 \\
\hline
ToStringInt & 0.0\% & 0 \\
\hline
ToStringIntMeasure & 0.0\% & 0 \\
\hline
addError & 0.0\% & 0 \\
\hline
addError & 0.0\% & 0 \\
\hline
addFailure & 0.0\% & 0 \\
\hline
addFailure & 0.0\% & 0 \\
\hline
addListener & 0.0\% & 0 \\
\hline
addTest & 100.0\% & 7 \\
\hline
assertFalse & 40.0\% & 9 \\
\hline
assertFalse:130 & 0.0\% & 0 \\
\hline
assertTrue & 50.0\% & 45 \\
\hline
assertTrue:122 & 0.0\% & 0 \\
\hline
collectTests & 0.0\% & 0 \\
\hline
createResult & 0.0\% & 0 \\
\hline
createTests & 0.0\% & 0 \\
\hline
endTest & 50.0\% & 3 \\
\hline
endTest & 0.0\% & 0 \\
\hline
errorCount & 0.0\% & 0 \\
\hline
exitListener & 0.0\% & 0 \\
\hline
fail & 0.0\% & 0 \\
\hline
fail:137 & 0.0\% & 0 \\
\hline
failureCount & 0.0\% & 0 \\
\hline
getMessage & 0.0\% & 0 \\
\hline
getName & 0.0\% & 0 \\
\hline
getTestMethodNamed & 0.0\% & 0 \\
\hline
hasMessage & 0.0\% & 0 \\
\hline
initListener & 0.0\% & 0 \\
\hline
reflectionRunTest & 0.0\% & 0 \\
\hline
removeListener & 0.0\% & 0 \\
\hline
run & 100.0\% & 4 \\
\hline
run & 0.0\% & 0 \\
\hline
run & 0.0\% & 0 \\
\hline
run & 0.0\% & 0 \\
\hline
run & 100.0\% & 1 \\
\hline
run:191 & 100.0\% & 8 \\
\hline
run:339 & 0.0\% & 0 \\
\hline
runBare & 50.0\% & 4 \\
\hline
runCount & 0.0\% & 0 \\
\hline
runOnly & 0.0\% & 0 \\
\hline
runOnly & 0.0\% & 0 \\
\hline
runOnly & 0.0\% & 0 \\
\hline
runProtected & 13.3\% & 4 \\
\hline
runTest & 0.0\% & 0 \\
\hline
setName & 100.0\% & 1 \\
\hline
setUp & 100.0\% & 4 \\
\hline
startTest & 0.0\% & 0 \\
\hline
startTest & 75.0\% & 4 \\
\hline
tearDown & 100.0\% & 3 \\
\hline
testAt & 0.0\% & 0 \\
\hline
testCount & 0.0\% & 0 \\
\hline
tests & 0.0\% & 0 \\
\hline
throw & 0.0\% & 0 \\
\hline
toString & 0.0\% & 0 \\
\hline
wasSuccessful & 0.0\% & 0 \\
\hline
\hline
VDMUnit.vdmpp & 12.9\% & 160 \\
\hline
\end{longtable}
\end{document}
